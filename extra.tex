\section{Free algebras}
\begin{defn}
A \emph{free associative algebra} \(k\langle x_{1},\ldots ,x_{n}\rangle \) of \(n\) non-commutating variables \(x_{1},\ldots ,x_{n}\) is the algebra with as \(k\)-basis the set of words \(w=x_{i_{1}}\ldots x_{i_N}\) in the alphabet \(x_{1},\ldots ,x_{n}\) including the empty word, denoted by \(1\).

Multiplication of these basis element is defined by the concatenation of words.
\end{defn}

\begin{defn}

If \(f_{1},\ldots ,f_{m}\) are elements of the free algebra \(k\langle x_{1},\ldots ,x_{n}\rangle \), we say that the algebra
\[k\langle x_{1},\ldots ,x_{n}\rangle /(f_{1},\ldots ,f_{m})\]
is \emph{generated} by \(x_{1},\ldots ,x_{n}\) with defining relations \(f_{1}=0,\ldots ,f_{m}=0\).
\end{defn}

\begin{thm}
Let \(V\) be a vector space, and let \(M_{i}\in \End_{k}(V)\) for \(i=1,\ldots ,n\). We can assign to a word \(w=x_{i_{1}}\cdots x_{i_N}\in k\langle x_{1},\ldots ,x_{n}\rangle \) the endomorphism
%
\[M_{w}=M_{i_{1}}\cdots M_{i_N}\in \End_{k}(V).\]
%
If we extend this linearly to \(k\langle x_{1},\ldots ,x_{n}\rangle \), we obtain a homomorphism \(\rho =\rho _{(M_{1},\ldots ,M_{n})} : k\langle x_{1},\ldots ,x_{n}\rangle \to \End_k (V)\). 

Note that this makes $(V,\rho)$ a representation of \(k\langle x_{1},\ldots ,x_{n}\rangle \).
\end{thm}

\begin{proof}
We need to show prove that \(\rho :k\langle x_{1},\ldots ,x_{n}\rangle \rightarrow \End_{k}(V)\) is a homomorphism. Showing that is equivalent with:
\begin{align*}\rho (w_{1})\rho (w_{2})&=\rho (w_{1}w_{2}) && \forall w_{1},w_{2}\in \mathcal{B}. \\
&\Updownarrow  \\
M_{w_{1}}M_{w_{2}}&=M_{w_{1}w_{2}} && \forall w_{1},w_{2}\in \mathcal{B}
\end{align*}

That this last statement is true follows directly from the definition of \(M_{w}\).
\end{proof}

\begin{thm}
Conversely, and representation of \(k\langle x_{1},\ldots ,x_{n}\rangle \) on \(V\) is of this form. There exists a natural bijection between the set of reprsentations of \(k\langle x_{1},\ldots ,x_{n}\rangle \) on \(V\) and the set of ordered \(n\) tuples of endormorphisms \((M_{1},\ldots ,M_{n})\) on \(V\).
\end{thm}

\begin{thm}
Let \(B\) be a finitely generated algebra, with generators \(y_{1},\ldots ,y_{n}\). Then there exists a unique surjective algebra homomorphism \(\phi :k\langle x_{1},\ldots ,x_{n}\rangle \rightarrow B\) such that \(\phi (x_{i})=y_{i}\). Assume that the two-sided ideal \(\Ker\phi \) is finitely generated, \(\Ker \phi =(f_{1},\ldots ,f_{n})\). Then
\[\bar{\phi }: k\langle x_{1},\ldots ,x_{n}\rangle  / (f_{1},\ldots ,f_{N} )\rightarrow B\]
is an isomorphism of algebras. We say that \(B\) has a presentation with finitely many generators and relations, or simply a finite presentation.

In this case, amy representation of \(B\) on \(V\) lifts to a representation of \(A\), and hence determines an \(n\)-tuple of endomorphisms \((M_{1},\ldots ,M_{n})\) of \(V\).

\end{thm}

\begin{thm}
The representation of \(k\langle x_{1},\ldots ,x_{n}\rangle  \) defined by an \(n\)-tuple of endomorphisms \((M_{1},\ldots ,M_{n})\) of \(V\) descends to \(B\) if and only if \(f_{i}(M_{1},\ldots ,M_{n})=0\) for all \(i=1,\ldots ,N\).
\end{thm}

\newpage

\section{Path algebras}

\begin{defn}
A \emph{quiver} \(Q=(I,E)\) is a finite directed graph, possibly with self loops and multiple edges. Here \(I\) denotes the set of vertices, and \(E\) the set of edges.

These sets come equipped with maps \(h:E\rightarrow I\) (head) and \(t:E\rightarrow I\) (tail). Given \(e\in E\), we write \(e':=t(e)\in I\) and \(e'':=h(e)\in I\).

\end{defn}

\begin{defn}
The \emph{\(n\)-loop quiver} \(Q_{L,n}\) is the quiver with one vertex \(v\) and \(n\) edges \(e_{1},\ldots ,e_{n}\).
\end{defn}

\begin{defn}
The Dynkin quiver \(Q(A_{n})\) of type \(A_{n}\) is the quiver with \(n\)vertices \(v_{1},\ldots ,v_{n}\) and \(n-1\) edges \(e_{1,2},e_{2,3},\ldots ,e_{n-1,n}\) such that
\begin{gather*}t(e_{i,i+1})=v_{i+1} \\
h(e_{i,i+1}) =v_{i}\end{gather*}

\end{defn}

\begin{defn}
A representation \(\mathcal{V}\) of a quiver \(Q=(I,E)\) consists of a \(k\)-vector space \(V_{i}\) for every \(i\in I\), and a \(k\)-linear map \(x_{h}: V_{h'}\rightarrow V_{h''}\) for every \(h\in E\).
\end{defn}


\begin{defn}
Let \(\mathcal{V}=(V_{i},x_{h})\) and \(\mathcal{W}=(W_{i},y_{h})\) be representations of the quiver \(Q\). A homomorphism \(\phi : \mathcal{V}\rightarrow \mathcal{W}\) of quiver representations consist of a collection of \(k\)-linear maps \(\phi _{i}:V_{i}\rightarrow W_{i}\) such that for all \(h\in E\):
\[y_{h}=\varphi _{h'}=\varphi _{h''} \circ  x_{h}\]

\end{defn}

\begin{defn}

The \emph{path algebra} \(P_{Q}\) of a quiver \(Q\) over a field \(k\) is the \(k\)-vector space with basis consisting of the oriented paths in \(Q\), including, for each \(i\in I\), the trivial path \(p_{i}\). The multiplication of basis elements in \(P_{Q}\) is defined by concatenation of paths if the paths are compasable, i.e. if path $a$ starts in the end point of path \(b\). In this case we defined \(ab\) to be the path where we first traverse \(b\) and then \(a\). If \(a\) and \(b\) are not composable, then we define \(ab=0\). We extend this bilinearly. This defines an associative \(k\)-algebra \(P_{Q}\) with unit $1=\sum _{i=1}p_{i}$.

\end{defn}

\begin{prop}
Show that \(\sum _{i\in I}p_{i}=1\).
\end{prop}

\begin{proof}

Note that \(1\) is the \textbf{unique} element of algebra \(P_{Q}\) such that:
\begin{align*}a\cdot 1=a=1\cdot a && a\in P_Q.\end{align*}
So we need to show that \(\sum _{i\in I}p_{i}=1\) satisfies the property. Let \(a\in A\), then \(a\) is some path that starts in \(j\) and in \(k\). Therefore:

\begin{gather*}a\sum _{i\in I}p_{i}=ap_{j}=a \\
\sum _{i\in I}p_{i}\; a=ap_{k}=a.\end{gather*}
\end{proof}

\begin{thm}
Show that the algebra \(P_{Q}\) is generated by \(p_{i}\) for \(i\in I\) and \(a_{h}\) for \(h\in E\) with the following defining relations:

\begin{enumerate}
  \item \(p_{i}^{2} =p_{i}\)
  \item \(p_{i}p_{j}=0 \quad \text{for } i\neq j \)
  \item \(a_{h}p_{h'}=a_{h}\)
  \item \(a_{h}p_{j}=0\quad \text{for }j\neq h'\)
  \item \(p_{h''}a_{h}=a_{h}\)
  \item \(p_{i} a_{h}=0 \quad \text{for } i\neq h''\)
\end{enumerate}

\end{thm}

\begin{prop}
Finite dimensional representations of \(P_{Q}\) and of \(Q\) bijectively correspond to each other, via the following correspondence.

If \((V,\rho )\) is a finite dimensional represenation of \(P_{Q}\), then we put \(V_{i}=\rho (p_{i})(V)\) for all \(i\in I\).

Since \(\{\rho (p_{i})\}_{i\in I}\) is a complete set of mutually ortogonal idempotents, we have a corresponding direct sum decomposition ( which is also a direct product) \(V=\oplus _{i\in I}\)

\end{prop}
\newpage
\section{Semisimple representations}

\begin{defn}

A \emph{semisimple} (or \emph{complete reducible}) representation of an algebra \(A\) is a direct sum of irreducible representations.

\end{defn}

\begin{prop}

Let \(V\) be an irreducible representation of \(A\) of dimension \(n\). Then \(\End (V)\) is a semisimple representation of \(A\). Where the action of \(A\) on $\End(V)$ is defined as:
\[a.L:V\rightarrow V:v\mapsto a.L(v)\]

\end{prop}

\begin{proof}
We are going to show that
\[\End(V) \cong  \underbrace{V \oplus  \cdots \oplus V}_{n-\text{times}}(:=nV).\]

We need to construct an isomorphism of representations. Let \(\mathcal{V}:=\{v_{1},\ldots ,v_{n}\}\) be a basis of \(V\). Define:
\[f:\End(V) \rightarrow nV : L  \mapsto \, \Big(L(v_{1}),\ldots ,L(v_{n})\Big).\]
\emph{Surjectivity:} \\Let \(y\in nV\). Then \(y=(y_{1},\ldots ,y_{n})\). Define \(L\in \End(V)\) such that \(L(v_{1})=y_{1},\ldots ,L(v_{n})=(y_{n})\).

\emph{Injectivity:} \\Let \(f(L)=f(K)\). Then
\[\, \Big(L(v_{1}),\ldots ,L(v_{n})\Big)=\Big(K(v_{1}),\ldots ,K(v_{n})\Big)\]
Then:
\begin{align*}L(v_{i})=K(v_{i}) && \forall i=1,\ldots ,n\end{align*}
Then \(L=K\).
\newpage
\emph{Homomorphism}: \\We need to show that:
\begin{align*}f(a.L)=a.f(L) && \forall a\in A,\forall L\in \End(V)\end{align*}
We have:
\[f(a.L)=\, \Big(a.L(v_{1}),\ldots ,a.L(v_{n})\Big)=a.f(L).\]
\end{proof}


\begin{prop}

Any semisimple representation \(V\) of \(A\) is canonically identified with
\[\bigoplus_{X} \Hom_{A} (X,V) \otimes  X\]
where \(X\) runs over all irreducible representations of \(A\).
\end{prop}

\begin{proof}

Define the map:
\[f:\bigoplus_{X} \Hom(X,V) \otimes X \rightarrow V : g\otimes x \mapsto  g(x).\]

Or more formally:
\[f:\bigoplus_{X} \Hom(X,V) \otimes X \rightarrow V : (g_{i}\otimes x_{i})_{i\in I}\mapsto \sum _{i\in I}g_{i}(x_{i}).\]

\emph{Surjectivity:} Take an element of $v\in V$. As \(V\) is semi-simple, we have that \(V=\bigoplus_{Y} Y\) where \(Y\) runs over a subset of all irreducible representations.

\end{proof}

\newpage
\section{Week 39 Theory}

\begin{defn}

Let \(A\) be a \(k\)-algebra. We call \(a\in A\) an \emph{idempotent element} or simply an \emph{idempotent} if \(a^{2} =a\).
\end{defn}

\begin{defn}

Two idempotents \(a,b\in A\) are called ortogonal if \(ab=ba=0\).

\end{defn}

\begin{prop}

If \(a,b\) are ortogonal, then $a+b$ is an idempotent.

\end{prop}

\begin{proof}
We need to show that
\((a+b)^{2} =a+b\).

Note that:
\[a^{2} +2ab+b^{2} =a+ab+ba+b = a+b\]

\end{proof}

\begin{defn}
A set of mutually ortogonal idempotents \(a_{1},\ldots ,a_{n}\) of \(A\) is called complete if
\[1=a_{1}+\cdots +a_{n}.\]

\end{defn}

\begin{defn}

An idempotent \(a\in Z(A)\) is called a \emph{central idempotent} of \(A\).

\end{defn}

\begin{defn}

A nonzero idempotent \(a\in A\) is called \emph{minimal} if any decomposition \(a=b+c\) of \(a\) as a sum of two orthogonal idempotents \(b,c\) implies that \(b=0\) or \(c=0\).

\end{defn}

\begin{prop}
If $V$ is a $k$-vector space then $\End_{k}(V)$ is a $k$-algebra.
\end{prop}

\begin{proof}
To show that $\End_{k}(V)$ is a $k$-algebra, we need to show that there exists a ring homomorphism:
\[f : k\rightarrow Z\Big(\End_{k}(V)\Big).\]
Define it as follows:
\[f : k\rightarrow Z\Big(\End_{k}(V)\Big) : \alpha  \mapsto  \alpha I\]
We then have:
\begin{gather*}f(\alpha +\beta )=(\alpha +\beta )I=\alpha I+\beta I=f(\alpha )+f(\beta ) \\
f(\alpha \beta )=(\alpha \beta )I=\alpha I\beta I\end{gather*}
\end{proof}

\begin{defn}
Given a representation \(V\) of \(A\). Then \(A'=\End_{A} (V)\) is a \(k\)-algebra, and \(V\) is also a representation of \(\End_{A} (V)\), in that case \(V\) is called the \emph{centralizer module}. If we need to distinguish the two module structures on \(V\), we will write $_{A} V$ or $_{A'} V$

\end{defn}

\begin{prop}
If $V$ is a $k$-vector space then $\End_{k}(V)$ is a $k$-algebra. Explain this statement.
\end{prop}

\begin{proof}
To show that $\End_{k}(V)$ is a $k$-algebra, we need to show that there exists a ring homomorphism:
\[
f : k\rightarrow Z\Big(\End_{k}(V)\Big).
\]
Define it as follows:
\[
f : k\rightarrow Z\Big(\End_{k}(V)\Big) : \alpha  \mapsto  \alpha I
\]
To show that \(\alpha I\in Z\big(\End_{k}(V)\big)\):
\begin{align*}
\alpha I\circ L=L\circ \alpha I &&\forall L\in \End_{k}(V)
\end{align*}
We have a ringhomomorphism as:
\begin{gather*}
f(\alpha +\beta )=(\alpha +\beta )I=\alpha I+\beta I=f(\alpha )+f(\beta ) \\
f(\alpha \beta )=(\alpha \beta )I=\alpha I\beta I=f(\alpha )f(\beta )
\end{gather*}
\end{proof}

\begin{prop}
If $V$ is a representation of \(k\)-algebra \(A\), then \(\End_A(V)\) is a \(k\)-algebra.
\end{prop}

\begin{proof}
To show that $\End_A(V)$ is a \(k\)-algebra, we need to show that there exists a ring homomorphism:
\[
f : k\rightarrow Z\Big(\End_A(V)\Big).
\]
Define it as follows:
\[
f : k\rightarrow Z\Big(\End_A(V)\Big) : \alpha  \mapsto  \alpha .I
\]
To show that \(\alpha I\in Z\big(\End_{k}(V)\big)\):
\begin{align*}
\alpha I\circ L=L\circ \alpha I &&\forall L\in \End_{k}(V)
\end{align*}
We have a ringhomomorphism as:
\begin{gather*}
f(\alpha +\beta )=(\alpha +\beta ).I=\alpha .I+\beta .I=f(\alpha )+f(\beta ) \\
f(\alpha \beta )=(\alpha \beta ).I=\alpha .I\beta .I=f(\alpha )f(\beta )
\end{gather*}
\end{proof}

\begin{prop}
To show that \(\End_A(V)\subseteq \End_{k}(V)\)
\end{prop}

\begin{proof}
An element \(\psi \in \End_A(V)\) is a linear map with an extra property. That
\[
a.\psi (v)=\psi (a.v)
\]
And this hold also for all elements \(a\in k\subseteq A\). Therefore \(\psi \) is a \(k\)-linear map.
\end{proof}


\begin{prop}
Given a representation \(V\) of \(A\). Then \(V\) is also a representation of \(A'=\End_{A} (V)\).

\end{prop}

\begin{proof}
We define:
\[\rho : A' \rightarrow  \End_{k} (V) : \phi  \mapsto  \phi \]
This is clearly a homomorphism of algebras.
\end{proof}

\begin{prop}
The image of an idempotent element under an algebra homomorphism is again an idempotent element.
\end{prop}

\begin{proof}
Let \(a\) an element in \(A\) such that \(a^{2} =a\). We need to show that
\[
f(a)^2=f(a).
\]
We have:
\[
f(a)^2=f(a)f(a)=f(a^2)=f(a).
\]
\end{proof}

\begin{prop}
For any idempotent \(e\in A\) we have that

\begin{enumerate}
  \item \(\rho (e)\in \End_{k}(V)\) is idempotent
  \item \(\rho (e)\in \End_A(V)\) if \(e\) is central
\end{enumerate}

\end{prop}

\begin{proof}

\begin{enumerate}
  \item We need to show that \(\rho (e)^2=\rho (e)\). This follows as \(\rho \) is an algebra homomorphism.
  \item Assume that \(e\) is central. Then
\begin{align*}
ae=ea &&\forall a\in A
\end{align*}
We need to show that
\[
\rho (e) \in \End_A(V).
\]
So we need to show that
\begin{align*}
a.\rho _{(e)}(v)=\rho _{(e)}(a.v) &&\forall a\in A,\forall v\in V
\end{align*}
In other notation:
\begin{align*}
a.e.v=(ae).v=(ea).v=e.a.v &&\forall a\in A,\forall v\in V
\end{align*}
\end{enumerate}

\end{proof}

\begin{prop}
For any \(x\in \End_A(V)\) the subspaces \(\Ker(x)\subset V\) and \(\Imm(x)\in V\) are \(A\)-submodules.
\end{prop}

\begin{proof}
To show that \(\Ker(x)\) is a subrepresentation of \(V\), we need to show that:
\begin{align*}
a.\Ker(x)\subseteq \Ker(x) &&\forall a\in A
\end{align*}
Let \(a\in A\) and let \(v\in \Ker(x)\). Then \(x(v)=0\). We want to show that
\[
a.v\in \Ker(x)
\]
which is equivalent with showing that
\[
x(a.v)=0.
\]
This hold as
\[
x(a.v)=a.x(v)=a.0=0.
\]

To show that \(\Imm(x)\) is a subrepresentation of \(V\), we need to show that:
\begin{align*}
a.\Imm(x)\subseteq \Imm(x) &&\forall a\in A
\end{align*}
Let \(a\in A\) and let \(w\in \Imm(x)\). We want to show that
\[
a.w\in \Imm(x).
\]
Which is equivalent with showing that \(\exists v\in V\) such that
\[
x(v)=a.w.
\]

We know that there exists a \(v'\in V\) such that \(x(v')=w\). So take \(v:=a.v'\). Then
\[
x(v)=x(a.v')=a.x(v')=a.w.
\]
\end{proof}
\newpage
\begin{prop}
If \(V=U\oplus W\) is a decomposition of \(_AV\) as direct sum of \(A\)-submodules. Then the projection \(e_{u}\) of \(V\) onto \(U\) along \(W\) and the projection \(e_{w}\) of \(V\) onto \(W\) along \(U\) satisfy:

\begin{enumerate}
  \item \(e_U,e_W \in \End_A(V)\)
  \item \(e_U\) and \(e_W\) are orthogonal idempotent elements
  \item \(1=e_U+e_W\)

\end{enumerate}
\end{prop}
\begin{proof}

\begin{enumerate}
  \item To show that \(e_U\in \End_A(V)\), we need to show that:
\[
a.e_U(v)=e_U(a.v)
\]
As \(V=U\oplus W\), we have that \(a.v=a.(u\oplus w)=a.u\oplus a.w\). If we regard \(U,V\) as subspaces of \(V\) with \(U\cap V=0\) we can write $a.v=a.u+aw$. Therefore:
\[
e_U(a.v)=e_U(a.u+a.w)=a.u=a.e_U(v)
\]
  \item \textbf{To show that \(e_U\) is idempotent}, we need to show that
\[
e_U \circ e_U=e_U.
\]
This is equivalent with showing that
\[
e_U\circ e_U(v)=e_U(v).
\]
Any \(v\) can be written as \(v=u=w\), and we have:
\begin{align*}
e_U(v)=u &&e_U\circ e_U(v)=e_U(u)=u.
\end{align*}
\textbf{To show that \(e_U\) and \(e_W\) are orthogonal}, we need to show that
\begin{align*}
e_U\circ e_W=0 && e_W\circ e_U=0.
\end{align*}
This is equivalent with showing that:
\begin{align*}
e_U\circ e_W(v)=0 && e_W\circ e_U(v)=0.
\end{align*}
This holds as:
\begin{align*}
e_U\circ e_W(v)=e_U(w)=0 &&e_W\circ e_U(v)=e_W(u)=0 
\end{align*}
  \item To show that \(e_U+e_W=1\), we need to show that:
\[
e_U+e_W(v)=1(v)
\]
This holds as:
\[
e_U+e_W(v)=e_U(v)+e_W(v)=u+w.
\]
\end{enumerate}

\end{proof}

\begin{prop}
If \(e\in \End_A(V)\) is an idempotent element, then \(1-e\in A'\) is an idempotent as well.
\end{prop}

\begin{proof}
To show that \(1-e\) is an idempotent element, we need to show that:
\[
(1-e)^2=1-e
\]
As $e^2=e$, we have that:
\[
(1-e)^2=1-2e+e^2=1-2e+e=1-e.
\]
\end{proof}

\begin{prop}
Show that \(e\) and \(1-e\) are orthogonal.
\end{prop}

\begin{proof}
To show that \(e\) and \(1-e\) are orthogonal, we have to show that:
\begin{align*}
e(1-e)=0 &&(1-e)e=0.
\end{align*}
This holds as
\begin{gather*}
e(1-e)=e-e^2=e-e=0 \\
(1-e)e=e-e^2=e-e=0
\end{gather*}
\end{proof}

\begin{prop}
Show that \(V=eV\oplus (1-e)V\) is a decomposition of \(_AV\) as direct sum of \(A\)-submodules.
\end{prop}

\begin{proof}
To show that \(V=eV\oplus (1-e)V\) we need to show that:

\begin{enumerate}
  \item \(eV \cap  (1-e)V =0\)
  \item \(eV + (1-e)V = V\)
\end{enumerate}

\textbf{To show that \(e.V \cap  (1-e).V =0\)}, we need to show that:
\begin{align*}
v\in e.V\space  \wedge  \space v\in (1-e).V && \Longrightarrow  &&v=0
\end{align*}
Note that the action of \(e\) on \(V\) is defined as \(e.v=e(v)\). So we have that
\(v=e(v')\) for some $v'$ in \(V\). And also that \(v=(1-e)(v'')\) for some $v''\in V$. So we have:
\begin{gather*}
(1-e)(v'')=e(v') \\
\Downarrow \\
v'' =e(v'')+e(v')\\
\Downarrow \\
e(v'')=e(v'')+e(v') \\
\Downarrow \\
e(v')=0 \\
\Downarrow \\
v=0
\end{gather*}
\textbf{To show that \(e.V+(1-e).V=V\)}, we need to show that any $v\in V$ can be written as \(e(v')+v''-e(v'')\) for some $v'$ and $v''$ in $V$.

This holds, if we take $v'=:v'':=v$.
\end{proof}


\begin{prop}
Show that $_AV$ is an indecomposable module iff $1\in \End_A(V)$ is a minimal idempotent.
\end{prop}

\begin{proof}
To show that \(_AV\) is an indecomposable module, we need to assume that
\(_AV=U\oplus W\) and show that either $U$ or $W$ is zero.

Then $1=e_U+e_W$ with $e_U$ and $e_W$ orthogonal idempotents elements. But as $1$ is minimal, we must have that $e_U$ or $e_W$ is zero. Say that $e_U=0$. This means that $u=0$, for any $u$ in $U$. And so $U=0$.

To show that $1\in  \End_A(V)$ is a minimal idempotent, it satisfies to assume that $1=b+c$ where $b,c$ are orthogonal idempotents and show that either $b$ or $c$ is zero.

We know that \(V=bV\oplus cV\) must be a decomposition of \(_AV\). But as \(_AV\) is indecomposable, either \(bV\) or \(cV\) must be zero. That means exactly that \(b(v)=0\) or \(c(v)=0\) for all \(v\in V\). Which gives us that \(b\) or \(c\) is zero.
\end{proof}

\begin{prop}[Exercise 1 (Week 39)]
Let \(V\) be an \(A\) module, and suppose that \(e_{1},\ldots ,e_{n}\) is a complete set of orthogonal idempotents of \(\End_A(V)\). Show that \(V=\bigoplus_{i=0}^ne_{i}V\) is a decomposition of \(V\) as direct sum of \(A\)-submodules.
\end{prop}

\begin{proof}
Note that the action of \(A'\) on \(V\) is defined as \(\phi .v=\phi (v)\). So showing that \(e_{i}.V\) is a subrepresentation of \(V\), is equivalent with showing that \(\Imm(e_{i})\) is a subrepresentation of \(V\), which holds for any \(x\in \End_A(V)\).

To show that \(e_{i}V\cap e_{j}V=0\), we need to show that
\begin{align*}
v\in \Imm(e_{i})\cap v\in \Imm(e_{j}) &&\Longrightarrow &&v=0.
\end{align*}
Assume \(v=e_{i}(v')=e_{j}(v'')\) for some \(v',v''\in V\). We know that \(e_{i}e_{j}=0\) and \(e_{i}e_{i}=e_{i}\). So we have:
\begin{gather*}
e_{i}e_{i}(v')=e_{i}e_{j}(v'') \\
\Downarrow  \\
e_{i}(v')=0(v'') \\
\Downarrow  \\
v=0
\end{gather*}
To show that \(V\subseteq \bigoplus_{i=0}^ne_{i}V\), we assume that \(v\in V\) and show that
\begin{align*}
v=e_{1}v_{1}+\ldots +e_{n}v_{n} &&\exists v_{1},\ldots ,v_{n}\in V.
\end{align*}
Take \(v_{1}:=\ldots :=v_{n}:=v\), then
\[
e_{1}v_{1}+\ldots +e_{n}v_{n}= (e_{1}+\ldots +e_{n})v=v.
\]
\end{proof}


\begin{prop}
Let \(U,V\) be finite dimensional \(k\)-vector spaces with direct sum decompositions \(U=\bigoplus_{j=1}^nU_{j}\) and \(V=\bigoplus_{i=1}^mV_{i}\) and with corresponding complete sets of idempotents \(e_{j}\in \End_{k}(U)\) and \(f_{j}\in \End_{k}(V)\).

We have a \(k\)-linear isomorphism \(M\) between \(\Hom_{k}(U,V)\) and the \(k\)-vector space consisting of \(m\times n\) matrices

\[\begin{pmatrix}\phi _{1,1} &\cdots &\phi _{1,n} \\
\vdots  & & \vdots  \\
\phi _{m,1} &\cdots &\phi _{m.n} \end{pmatrix}\]

with \(\phi _{i,j}\in \Hom_{k}(U_{j},V_{i})\). This isomorphism \(M\) is defined by \(\phi  \mapsto  (\phi _{i,j})\) where \(f_{i}\phi |_{U_{j}}\).
\end{prop}

\begin{prop}
If we decompose \(u=\sum u_{j}\) with \(u_{j}\in U_{j}\), and \(\phi (u)=\sum \phi (u)_{i}\) with \(\phi (u)_{i}\in V_{i}\) we have:
\begin{align*}
\begin{pmatrix}\phi (u)_{1} \\ \vdots  \\ \phi (u)_{m}\end{pmatrix} =
\begin{pmatrix}\phi _{1,1} & \cdots  & \phi _{1,n}\\
\vdots  & & \vdots  \\
\phi _{m,1} & \cdots  & \phi _{m,n}\end{pmatrix}
\begin{pmatrix}u_{1} \\ \vdots  \\ u_{n}\end{pmatrix}
\end{align*}
\end{prop}

\begin{proof}
Let's look at the ith row, we need to show that
\[
\phi _{i,1}u_{1}+\ldots +\phi _{i,n}u_{n}=\phi (u)_{i},
\]
which is equivalent with showing that,
\[
f_{i}\phi (u_{1})+\cdots +f_{i}\phi (u_{n})=\phi (u)_{i},
\]
which is equivalent with showing that,
\[
f_{i}\phi (u)=\phi (u)_{i}.
\]
It is clear that both elements are in \(V_{i}\) as \(V_{i}=f_{i}V\). That the elements must be equal follows out of the uniqueness of the decomposition
\[
\phi (u)=\sum \phi (u)_{i}.
\]
\end{proof}

\begin{prop}
If \(W=\bigoplus_{l=1}^p W_{l}\) is a direct sum of finite dimensional \(k\)-modules, and if \(\psi \in \Hom_{k}(V,W)\), then
\[
M(\psi \circ \phi )=M(\psi )\circ M(\phi ).
\]
\end{prop}

\begin{proof}
First remember that a \(k\)-module is indentical to a \(k\)-vector space. ... 
\end{proof}


\begin{prop}
Let \(U\), \(V\) be finite dimensional \(A\)-modules with direct sum decomposition
\begin{align*}
U=\bigoplus_{j=1}^nU_{j} && && V=\bigoplus_{i=1}^mV_{i}
\end{align*}
as \(A\)-modules with corresponding complete sets of orthogonal idempotents \(e_{j}\in \End_A(U)\) and \(f_{i}\in \End_A(V)\).

Given \(\phi \in \Hom_{k}(U,V)\) with \(M(\phi )=(\phi _{i,j})\) we have
\begin{align*}
\phi \in \Hom_A(U,V) \Longleftrightarrow  \phi _{i,j}\in \Hom_A(U_{j},V_{i}) &&\forall i,j
\end{align*}
\end{prop}

\begin{proof}
To show that \(\phi \in \Hom_A(U,V)\), it suffices to show that
\begin{align*}
\phi \circ \rho _{a}(u)=\rho '_{a}\circ \phi (u) &&\forall a\in A,\forall u\in U.
\end{align*}
Fix \(a\in A\), and write \(\rho =\rho (a)\). As $M$ is an isomorphism, it suffices to show that
\begin{align*}
(\phi _{i,j})\circ (\rho _{i,j})(u)=(\rho '_{i,j})\circ (\phi _{i,j})(u) &&\forall u\in U,
\end{align*}
which is equivalent with showing that,
\[
\phi _{i,1}\rho _{i,1}u_{1}+\ldots +\phi _{i,n}\rho _{i,n}u_{n}=\rho _{i,1}\phi _{i,1}u_{1}+\ldots +\rho _{i,n}\phi _{i,n}u_{n}.
\]

This hold as:
\begin{align*}
\phi _{i,j}\in \Hom_A(U_{j},V_{i}) &&\forall i,j
\end{align*}
\end{proof}

\begin{prop}
Given \(U=\oplus _{j=1}^nU_{j}\), a direct decomposition of \(A\)-modules, \(M\) defines an isomorphism between \(\End_A(U)\) and the algebra of \(n\times n\) matrices \((\phi _{i,j})\) where \(\phi _{i,j}\in \Hom_A(U_{j},U_{i})\).

In particular, if \(V\) is an \(A\)-module, \(U=V^n\), then \(M\) defines an isomorphism of \(\End_A(V^n)\) with the algebra \(\Mat_{n}(\End_A(V))\).
\end{prop}

\begin{defn}
An \(A\)-module \(V\) is called \emph{semisimple} if \(V\) can be decomposed as a direct sum of irreducible submodules.
\end{defn}

\begin{defn}
Let \(\Irr(A)\) denote a complete set of representatives for the equivalence classes of irreducible representations of \(A\).
\end{defn}

\begin{prop}
Let \(V=V_{1}\oplus \cdots \oplus V_{n}\) be a finite dimensional semisimple \(A\)-module. For each \(U\in \Irr(A)\) let
\[
n_U:=\dim_{k} (\Hom_A(U,V)) \in  \Bbb{Z}_{\geq 0}.
\]
Show that \(n_U\) is equal to the number of \(V_{i}\) that are equivalent to \(U\).
\end{prop}

\begin{proof}
Note that by Prop 1.5, \(\dim_{k}(\Hom_A(U,V))=\sum _{i}\dim(\Hom_A(U,V_{i})\).

Remember that if \(V_{1}\) and \(V_{2}\) are inequivalent irreducible \(A\)-modules, then \(\Hom_A(V_{1},V_{2})=0\). Therefore \(\Hom_A(U,V_{i})=0\) always except it \(U\) is equivalent with \(V_{1},\ldots ,V_{l}\). In that case
\[
\Hom_A(U,V_{i})\cong \End_A(U)\cong k.\Id_U.
\]
\end{proof}

\begin{prop}
Let \(V\) be a finite dimensional semisimple \(A\)-module. For each \(U\in \Irr(A)\) let
\[
n_U:=\dim_{k} (\Hom_A(U,V) \in  \Bbb{Z}_{\geq 0}.
\]
Then \(\{U\in \Irr(A) : n_U\neq 0\}=\{U_{1},\ldots ,U_{l}\}\) is a finite set, and we have:
\begin{gather*}
V\cong n_{1}U_{1}\oplus \ldots \oplus n_{l}U_{l} \\
\\
\End_A(V)\cong \Mat_{n_{1}}(k)\oplus \cdots \oplus \Mat_{n_{l}}(k)
\end{gather*}
\end{prop}

\begin{proof}
To show that \(V\cong n_{1}U_{1}\oplus \cdots \oplus n_{l}U_{l}\), it suffices to show that there exists an isomorphism of representations
\[
\phi :V\rightarrow n_{1}U_{1}\oplus \cdots \oplus n_{l}U_{l},
\]
which in turn is equivalent with showing that \(\exists \phi \) such that:

\begin{enumerate}
  \item \(\phi \in \Hom_A(V,n_{1}U_{1}\oplus \cdots \oplus n_{l}U_{l})\)
  \item \(\phi \) is invertible
\end{enumerate}

Let \(j:\{1,\ldots ,N\}\rightarrow \{1,\ldots ,l\}\) be the function such that \(V_{i}\) is equivalent with to \(U_{j(i)}\). By Schur's lemma, there exist \(A\)-module isomorphism \(\phi _{i,i}\in \Hom_A(V_{i},U_{j(i)})\). Define \(\phi _{i,k}=0\) for \(i\neq k\). We have now that,
\begin{align*}
\phi _{i,k}\in \Hom_A(V_{k},U_{j(i)}) &&\forall i,k,
\end{align*}
which gives us that the \(N\times N\) matrix \((\phi _{i,j})\) corresponds to a
\[
\phi \in \Hom_A(V,n_{1}U_{1}\oplus \cdots \oplus n_{l}U_{l}).
\]
Choose \(\phi \) in that way.

To show that \(\phi \) is invertible, note that \((\phi _{i,j})\) is diagonal, and the elements on the diagonal are invertible.
\end{proof}


\begin{defn}
Let \(U\) be a subrepresenation of \(V\), a \emph{complement} \(W\) of \(V\) is a subrepresentation such that \(V=U\oplus W\).
\end{defn}

\begin{prop}
Let \(A\) be a \(k\)-algebra, and let \(V\) be a finite dimensional (over \(k\)) \(A\)-module. The following are equivalent:

\begin{enumerate}
  \item \(V\) is a sum of irreducible submodules
  \item \(V\) is semisimple
  \item Every submodule \(U\subseteq V\) admits a complement.
\end{enumerate}
\end{prop}


\begin{proof}
[Proof of \(1\Rightarrow 2\)] Showing that \(V\) is semisimple, is equivalent with showing that there exist a \textbf{direct} sum of irreducible submodules equal to \(V\).

By hypothesis, we already have a sum \(\sum _{i\in I}V_{i}\) of irreducible submodules equal to \(V\). As \(V\) is finite, we can choose a finite subset of submodules \(V_{i}\) such that \(\sum _{i=1}^nV_{i}=V\). We can reduce our problem now to showing that \(V_{i}\cap V_{j}=0\) for \(i\neq j\).

Assume that there would exist an element
\[
v\in V: v\in V_{i}\wedge v\in V_{j}.
\]
Then \(V_{i}=Av=V_{j}\). So \(V_{i}=V_{j}\).<br />
\end{proof}

\begin{proof}
[Proof of \(2 \Rightarrow 3\)] To show that every submodule \(U\subseteq V\) admits a complement, it suffices to assume that \(U\) is a submodule and show that there exists a submodule \(W\) such that \(V=U\oplus W\).

By hypothesis, we have a direct sum of irreducible submodules
\[
V=\bigoplus_{i=1}^nV_{i}.
\]
Let \(S_J:=U+\bigoplus_{i=1}^kV_{i}\) where $k$ is the maximal number such that $S_J$ is a direct sum. For any \(i : k<i\leq n\) we have that \(V_{i}\subseteq S_J\), since otherwise \(V_{i}\cap S_J=0\), contradicting the choice of $k$.
\end{proof}

\begin{proof}[Proof of \(3\Rightarrow 2\)] 
We are going to show that every \(A\)-module \(V\) with \(\dim(V)\leq k\), is a sum of irreducible submodules, by induction on \(k\).

\begin{itemize}
  \item \textbf{Basis case}: If \(k=1\), then \(V\) is irreducible, and \(V\) is a submodule of itself.
  \item \textbf{Induction step}: To show that every \(A\)-module \(V\) of $\dim(V)=k+1$ is a sum of irreducible submodules.
Take \(U\subseteq V\) a proper submodule. By hypothesis (3), we have that \(V=U\oplus W\). By induction hypothesis, we have that \(U,W\) are sums of irreducible submodules. And therefore \(V\) is a sum of irreducible submodules as well.
\end{itemize}
\end{proof}

\begin{prop}
Let \(V\) be a finite dimensional semisimple \(A\) module, and let \(U\subseteq V\) be a submodule. Then \(U\) and \(V/U\) are also semisimple.
\end{prop}

\begin{proof}
Showing that \(U\) is semisimple, is equivalent with showing that any submodule \(U'\subseteq U\) admits a complement \(W'\subseteq U\).

By hypothesis, we have that \(V\) is semisimple, and so \(U'\) as submodule of \(V\) has an complement \(W\subseteq V\). Define now \(W'=U\cap W\). We need to show that :

\begin{enumerate}
  \item \(U'\cap W' =0\)
  \item \(W'\) is a submodule of \(U\)
  \item \(U'+W'=U\)
\end{enumerate}

We will proof it one at a time:

\begin{enumerate}
  \item \(U'\cap W'=U'\cap (U\cap W)=U'\cap W=0\)
  \item Showing that \(W'\) is a submodule of \(U\), is equiv. with showing that,
\begin{align*}
a.W'\subseteq W' && \forall a\in A,
\end{align*}
so it suffices to assume \(a\in A\) and \(w'\in W'\) and show that
\[
a.w'\in W',
\]
showing that is equivalent with assuming that \(w'\in U\) and \(w'\in W\) and showing that
\[
a.w'\in U \wedge  a.w'\in W.
\]
This holds as \(U\) and \(W\) are both \(A\)-submodules.
  \item To show that \(U'+W'=U\), it suffices to show that \(U'+W'\supseteq  U\), to show that it suffices to assume \(u\in U\) and show that there exists \(u'\in U'\) and \(w\in W'\) such that \(u=u'+w'\).  <p>
We have that \(U'+W=V\) and as any \(U\subseteq V\), we have that \(u=u'+w\) for some \(w\in W\). And so \(w\) also has to be in $U$, and therefore in $W'=U\cap W$.
\end{enumerate}

\end{proof}

\begin{prop}
Let \(V\) be a finite dimensional semisimple \(A\) module, and let \(U\subseteq V\) be a submodule. Then \(U\) and \(V/U\) are also semisimple.
\end{prop}

\begin{proof}
To show that \(V/U\) is semisimple, it suffices to show that there exists a complement of \(V/U\).

As \(U\) is semisimple, it has a complement \(W\), so that
\[
V/U=(U\oplus W)/U \cong W.
\]
Therefore \(V/U\) has \(U\) as a complement.
\end{proof}

\begin{prop}[Exercise 5 (Week 39)]
Let \(U\) be a submodule of a finite dimensional \(A\)-module. If \(U\) and \(V/U\) are semisimple, show that \(V\) is semisimple.
\end{prop}

\begin{proof}
Showing that \(V\) is semisimple, is equivalent with showing that \(V\) is the sum of irreducible submodules.

If \(U\) is semisimple, it has a complement \(W\), and we have:
\[
V/U=(U\oplus W)/U \cong W
\]
As both \(U\) and \(W\) are now semisimple, we have that \(U\) and \(W\) are sums of irreducible submodules. And therefore \(V\) is also a sum of irreducible submodules.
\end{proof}

\begin{defn}
Let \(k\) be a field. A finite dimensional \(k\)-algebra \(A\) is called semisimple if all its finite dimensional representations are semisimple.
\end{defn}

\begin{prop}
A finite dimensional \(k\)-algebra \(A\) is semisimple if and only if the left regular representation \((A,\rho )\) of \(A\) is semisimple.
\end{prop}

\begin{proof}
[Proof of \(\Longrightarrow \)] If \(A\) is semisimple, then \textbf{all} its finite dimensional representations are semi-simple. And as \(A\) is finite-dimensional, the regular representation is finite dimensional, and so \((A,\rho )\) is indeed semisimple.
\end{proof}

\begin{proof}
[Proof of \(\Longleftarrow \)] To show that \(A\) is semisimple, it suffices to assume that \(V\) is a finite dimensional represenation and show that it is semisimple.

We can choose a finite subset \(B\subseteq V\) such that \(V=AB\). Why ?
\end{proof}


\begin{prop}
Let \(A''=\End_{A'}(V)=\End_{\End_{A}(V)}(V)\) be the centralizer algebra of centralizer module \(_{A'}V\). Then \(A''\) is a \(k\)-algebra.
\end{prop}

\begin{proof}
Showing that \(A''\) is a \(k\)-algebra, is equivalent with showing that there exists a ring homomorphism
\[
f: k\rightarrow Z(A'')=Z(\End_{A'}(V)).
\]

Define
\[
f:k\rightarrow Z(A'')=Z(\End_{A'}(V)) : \alpha  \rightarrow  \alpha \underset{A}{.}I\underset{A'}{.}I',
\]

where \(I'\) is the unit of \(\End_{A'}(V)\).

Showing that \(\alpha .I.I'\in Z(\End_{A'}(V))\), is equivalent with showing that
\begin{align*}
\alpha .I.I'\circ L = L\circ \alpha .I.I' &&\forall L\in \End_{A'}(V),
\end{align*}
which is equivalent with assuming that \(L\in \End_{A'}(V)\) and showing that
\begin{align*}
\alpha .I.I'\circ L(v) = L\circ \alpha .I.I'(v) &&\forall v\in V.
\end{align*}
Which holds as \((\alpha .I).L(v)=L((\alpha .I).v)\).

We have a ringhomomorphism as:
\begin{gather*}
f(\alpha +\beta )=(\alpha +\beta ).I.I'=\alpha .I.I'+\beta .I.I'=f(\alpha )+f(\beta ) \\
f(\alpha \cdot \beta )=(\alpha \cdot \beta ).I.I'=\alpha .I.I'\cdot \beta .I.I'=f(\alpha )\cdot f(\beta ) 
\end{gather*}
\end{proof}

\newpage
\begin{prop}
Show that \(V\) is an \(A''\)-module.
\end{prop}

\begin{proof}
To show that \(V\) is an \(A''\)-module, it suffices to show that there exist a homomorphism of algebras:
\[
\rho '' : A'' \rightarrow  \End_{k}(V)
\]

Note that any element \(\phi \in A''\) is an homomorphism of representations:
\[
\phi : {_{A'}}V\rightarrow  {_{A'}}V
\]

Which is a linear opeartor such that:
\begin{align*}
\phi (a'.v)=a'.\phi (v) &&a'\in \End_A(V)
\end{align*}
But is this \(\phi \) also a \(k\)-linear operator ? Which means that
\begin{align*}
\phi (\alpha v)=\alpha \phi (v) &&\forall \alpha \in k
\end{align*}
This does hold, as the operator \(\alpha .I\in \End_A(V)\) does act the same was \(\alpha \in k\) does.
In other words we may just use the mapping:
\[
\rho '' : A'' \rightarrow  \End_{k}(V): \phi  \mapsto \phi .
\]
\end{proof}


\begin{prop}
Show that \(\rho (A)\subseteq A''\).
\end{prop}

\begin{proof}
Let \(\phi \in \rho (A)\). Then \(\phi =\rho (a)\) for some \(a\in A\).

To show that \(\phi \in A''\), we need to show that:
\begin{align*}
\phi (a'.v)=a'.\phi (v) &&\forall a'\in \End_A(V), \forall v\in V.
\end{align*}
Note that the action of \(A'\) on \(V\) is defined as \(a'.v=a'(v)\), so we need to show that:
\begin{align*}
\phi (a'(v))=a'(\phi (v)) &&\forall a'\in \End_A(V), \forall v\in V.
\end{align*}
This holds as for \(a'\in \End_A(V)\) we have that:
\begin{align*}
a'(\rho (a)(v))=\rho (a)(a'(v)) && \forall a\in A, \forall a\in V.
\end{align*}
\end{proof}

\begin{defn}
We say that \(_AV\) has the double centralizer property if \(\rho (A)=A''\).
\end{defn}

\begin{thm}
Let \(V\) be a finite dimensional semisimple \(A\)-module. Then \(V\) has the DCP.
\end{thm}

\begin{prop}[Exercise 7 (Week 39)]
Let \(V=k^{2}\) be a two dimensional \(k[x]\)-module such that
\begin{align*}
\rho (x)=\begin{pmatrix}0& 1 \\ 0 & 0\end{pmatrix}
\end{align*}
Does \(V\) satisfy DCP?
\end{prop}

\begin{proof}
Showing that \(V\) doesn't satisfy DCP, is equivalent with showing that
\[
\End_{A'}(V) \not\subseteq \rho (A).
\]
It suffices to find a \(\phi \in \End_{k}(V)\) such that for any \(f\in k[x]\) we have:
\[
\phi \neq \rho (f)
\]

First note that
\begin{align*}
\rho (x^2)=\rho (x)\rho (x)=\begin{pmatrix}0 & 1 \\ 0 & 0\end{pmatrix}\begin{pmatrix}0 & 1 \\ 0 & 0\end{pmatrix} = 0.
\end{align*}
So if we choose \(\phi \neq 0\), we have that for any \(f\) of degree higher than 1:
\[
\phi  \neq  0 =\rho (f).
\]
Suppose \(\rho \) is of degree \(\leq 1\). Say that \(f=a+bx\). Then

\begin{align*}
\rho (a+bx)&=\rho (a)+\rho (bx)\\ 
&=a\rho (1)+b\rho (1)\begin{pmatrix}0& 1 \\ 0 & 0\end{pmatrix}  \\
&=aI+\begin{pmatrix}0& b \\ 0 & 0\end{pmatrix}  \ \\
&= \begin{pmatrix}a& b \\ 0 & a\end{pmatrix}
\end{align*}
And we have that:
\begin{align*}
\rho (a+bx)(e_{1})&=ae_{1} \\
\rho (a+bx)(e_{2}) &=ae_{1}+be_{2}
\end{align*}
For any \(\phi \in \End_{A'}(V)\) we have that:
\begin{align*}
\phi (f'(e_{1}))=f'(\phi (e_{1}))&& \forall f'\in \End_{k[x]}(V).
\end{align*}
Suppose that our \(\phi \) is of the form \(\rho (a+bx)\) and let \(f'(e_{1})=e_{2}\). Then we have:
\[
ae_{1}+be_{2}= \phi (e_{2})=\phi (f'(e_{1}))=f'(\phi (e_{1}))=f'(ae_{1})=af'(e_{1})=ae_{2}
\]

Which is only possible if \(a=0\) and \(a=b\). But then \(\phi =0\) and we had chosen \(\phi \) to be nonzero.
\end{proof}

