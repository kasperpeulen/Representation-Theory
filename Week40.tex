\section{Week 40}

\subsection{Homework}

\begin{prop}[Exercise 7 (Week 39)]
Let \(V=k^{2}\) be a two dimensional \(k[x]\)-module such that
\begin{align*}
\rho (x)=\begin{pmatrix}0& 1 \\ 0 & 0\end{pmatrix}
\end{align*}
Does \(V\) satisfy DCP?
\end{prop}

\begin{proof}
Showing that \(V\) doesn't satisfy DCP, is equivalent with showing that
\[
\End_{A'}(V) \not\subseteq \rho (A).
\]
It suffices to find a \(\phi \in \End_{k}(V)\) such that for any \(f\in k[x]\) we have:
\[
\phi \neq \rho (f)
\]
First note that
\begin{align*}
\rho (x^2)=\rho (x)\rho (x)=\begin{pmatrix}0 & 1 \\ 0 & 0\end{pmatrix}\begin{pmatrix}0 & 1 \\ 0 & 0\end{pmatrix} = 0.
\end{align*}
So if we choose \(\phi \neq 0\), we have that for any \(f\) of degree higher than 1:
\[
\phi  \neq  0 =\rho (f).
\]
Suppose \(f\) is of degree \(\leq 1\). Say that \(f=a+bx\). Then
\begin{align*}
\rho (a+bx)&=\rho (a)+\rho (bx)\\ 
&=a\rho (1)+b\rho (1)\begin{pmatrix}0& 1 \\ 0 & 0\end{pmatrix}  \\
&=aI+\begin{pmatrix}0& b \\ 0 & 0\end{pmatrix}  \ \\
&= \begin{pmatrix}a& b \\ 0 & a\end{pmatrix}
\end{align*}
And we have that:
\begin{align*}
\rho (a+bx)(e_{1})&=ae_{1} \\
\rho (a+bx)(e_{2}) &=ae_{1}+be_{2}
\end{align*}
For any \(\phi \in \End_{A'}(V)\) we have that:
\begin{align*}
\phi (f'(e_{1}))=f'(\phi (e_{1}))&& \forall f'\in \End_{k[x]}(V).
\end{align*}
Suppose that our \(\phi \) is of the form \(\rho (a+bx)\) and let \(f'(e_{1})=e_{2}\). Then we have:
\[
ae_{1}+be_{2}= \phi (e_{2})=\phi (f'(e_{1}))=f'(\phi (e_{1}))=f'(ae_{1})=af'(e_{1})=ae_{2}
\]

Which is only possible if \(a=0\) and \(a=b\). But then \(\phi =0\) and we had chosen \(\phi \) to be nonzero.
\end{proof}

\begin{prop}[Exercise 7b (Week 39)]
Let \(V=A\), viewed as the regular representation. What is \(\End_A(V)\)? Does \(V\) has the DCP.
\end{prop}

\begin{proof}
We have shown that \(\End_A(A)\cong A^{\op}\) in a homework exercise.

Showing that \(V\) has the DCP, is equivalent with showing that
\[
\End_{A^{\op}}(A)\subseteq \rho (A),
\]

to show that it suffices to assume that \(\phi \in \End_{A^{op}}(A)\) and show that
\begin{align*}
\exists a\in A: \phi =\rho (a) && 
\end{align*}
which is equivalent with showing that
\begin{align*}
\exists a\in A: \phi (b)=ab.&&\forall b\in A.
\end{align*}
As \(\phi \in \End_{A^{op}}(A)\) we have that,
\begin{align*}
\phi (b.a)=b.\phi (a), &&\forall a\in A,\forall b\in B
\end{align*}
which is equivalent with,
\begin{align*}
\phi (ab)=\phi (a)b, &&\forall a\in A,\forall b\in B
\end{align*}
and therefore
\begin{align*}
\phi (b)=\phi (1\cdot b)=\phi (1)b. &&\forall b\in B
\end{align*}
Choose \(a:=\phi (1)\) and we have exactly what we wanted:
\begin{align*}
\exists a\in A: \phi (b)=ab && \forall b\in B
\end{align*}
\end{proof}

\subsection{Other exercises}

\begin{prop}[Exercise 5 (Week 39)]
Let \(U\) be a submodule of a finite dimensional \(A\)-module. If \(U\) and \(V/U\) are semisimple, show that \(V\) is semisimple.
\end{prop}

\begin{proof}
Showing that \(V\) is semisimple, is equivalent with showing that \(V\) is the sum of irreducible submodules.

If \(U\) is semisimple, it has a complement \(W\), and we have:
\[
V/U=(U\oplus W)/U \cong W
\]
As both \(U\) and \(W\) are now semisimple, we have that \(U\) and \(W\) are sums of irreducible submodules. And therefore \(V\) is also a sum of irreducible submodules.
\end{proof}



